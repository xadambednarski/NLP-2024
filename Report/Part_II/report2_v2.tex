\documentclass[12pt]{article}

\usepackage[utf8]{inputenc}
\usepackage[T1]{fontenc}
\usepackage[a4paper, margin=1in]{geometry}  
\usepackage{graphicx}
\usepackage{hyperref}

\title{Anotacja korpusów oraz osadzenia słów i tekstów\\Część II: Modele przestrzeni wektorowych}
\author{Autorzy: Oliwer Krupa, Adam Bednarski, Jan Masłowski, Łukasz Lenkiewicz}
\date{\today}

\begin{document}

\maketitle
\newpage

\tableofcontents
\newpage

\section{Word Embeddings}

\subsection{Osadzenia słów w anotowanym korpusie (Word2Vec i Fasttext)}
Zadanie polegało na wygenerowaniu osadzeń słów z wykorzystaniem co najmniej dwóch różnych modeli. Wybraliśmy następujące modele osadzeń:
\begin{itemize}
    \item \textbf{Word2Vec}:
    \begin{itemize}
        \item dla języka polskiego: \url{https://dsmodels.nlp.ipipan.waw.pl}
    \end{itemize}
    \item \textbf{Fasttext}:
    \begin{itemize}
        \item dla języka polskiego: \url{https://fasttext.cc/docs/en/crawl-vectors.html}
    \end{itemize}
\end{itemize}

\subsection{Wizualizacja danych za pomocą t-SNE}
Dane zostały zwizualizowane za pomocą techniki t-SNE, z wykorzystaniem interaktywnych wykresów. Każdy punkt na wykresie odpowiada słowu z anotacją. Po najechaniu na punkt można odczytać zarówno słowo, jak i przypisaną do niego etykietę.

\subsection{Porównanie k-najbardziej podobnych słów dla dwóch modeli}
Dla każdego z modeli (Word2Vec i Fasttext) wygenerowano listy k-najbardziej podobnych słów w oparciu o osadzenia zaanotowanych słów. Następnie porównano listy podobieństw, uwzględniając różnice w reprezentacjach przestrzeni wektorowych.

\begin{table}[ht]
\centering
\begin{tabular}{|c|c|}
\hline
\textbf{Word}     & \textbf{Similar Words}                                                                                                          \\ \hline
by                & By, Żęby, muc, Dáibhí, Żey                                                                                                      \\ \hline
kto               & Kto, ktoś, Ktoś, ktokolwiek, nikt                                                                                                \\ \hline
"                 & „, ”, \begin{CJK}{UTF8}{bsmi}相撲\end{CJK}, ', pt                                                                                \\ \hline
odbiorcow         & dostawców, inwestorów, kontrahentów, przedsiębiorców, sprzedawców                                                                \\ \hline
człowiek          & Człowiek, człowiek, człowiek., człowiek-, .Człowiek                                                                             \\ \hline
kojarzyć          & kojarzy, skojarzyć, kojarzył, kojarzą, kojarzyc                                                                                  \\ \hline
seks              & sex, seks., seksSeks, seks-, -seks                                                                                               \\ \hline
dziwić            & dziwić-, dziwić., dziwi, Dziwić, dziwic                                                                                           \\ \hline
fakt              & faktem, faktu, Fakt, zważywszy, Faktem                                                                                           \\ \hline
zdrajca           & Zdrajca, patriota, zdrajco, zdrajcą, sługus                                                                                       \\ \hline
faszysty          & faszystę, antyfaszysty, faszysta, nacjonalisty, nazisty                                                                          \\ \hline
Ryju              & Bzim, Quka, Gufi, Czizz, Ygor                                                                                                    \\ \hline
krystyna          & krystynaa, krystyny, marzena, zuzanna, krysia                                                                                    \\ \hline
but               & But, INVOKE, No-One, bucik, tired                                                                                                \\ \hline
\end{tabular}
\caption{Similar words in Fasttext}
\end{table}
\begin{table}[h!]
\centering
\begin{tabular}{|c|c|}
\hline
\textbf{Word} & \textbf{Similar Words} \\
\hline
by & aby, że by, więc, jeżeli, ażeby \\
\hline
kto & dlaczego, czemu, jeżeli, niech, jeżeli \\
\hline
odbiorców & klientów, dostawców, użytkowników, producentów, widzów \\
\hline
człowiek & osobnik, osoba, tubylec, kobieta, Polak \\
\hline
kojarzyć & utożsamiać, skojarzyć, identyfikować, wiązać, wywodzić \\
\hline
seks & sexu, masturbacja, sex, erotyka, prostytucja \\
\hline
dziwić & obawiać, denerwować, domyślać, łudzić, oburzać \\
\hline
fakt & stwierdzenie, twierdzenie, przeświadczenie, przypuszczenie, hipoteza \\
\hline
zdrajca & sprzedawczyk, kolaborant, faszysta, łajdak, kanalia \\
\hline
but & bucik, skarpeta, sandał, trzewik, spodnie \\
\hline
\end{tabular}
\caption{Similar words in Word2Vec}
\end{table}

\subsection{Dyskusja wyników}
Otrzymane wyniki pokazują różnice w sposobie, w jaki modele Word2Vec i Fasttext modelują przestrzeń wektorową. Obserwowane różnice w podobieństwach między słowami mogą wynikać z różnych metod treningu oraz charakterystyki korpusów użytych do trenowania modeli. Model z biblioteki fastText sugerował często różne warianty tych samych słów (nierzadko błędnie zapisanych). Jedną z możliwych przyczyn może być brak odpowiedniego preprocessingu zbioru danych uczących przed trenowaniem modelu. W przypadku Word2Vec generowane wyrazy były bardziej różnorodne i rzeczywiście pochodziły ze słownika jezyka polskiego.

\section{Text Embeddings}

\subsection{Osadzenia zdań / tekstów (Fasttext i TF-IDF)}
Dla osadzeń całych zdań i tekstów zastosowano dwa modele:
\begin{itemize}
    \item \textbf{Bert} z biblioteki flair.
    \item \textbf{TF-IDF} z biblioteki sklearn.
\end{itemize}

\subsection{Wizualizacja osadzeń tekstów za pomocą t-SNE}
Podobnie jak w przypadku osadzeń słów, zastosowano t-SNE do wizualizacji osadzeń tekstów, z odniesieniem do etykiet anotacji. Wykresy mają interaktywny charakter, umożliwiając użytkownikowi odczytanie pełnego zdania i jego anotacji po najechaniu na punkt.

\subsection{Klasteryzacja osadzeń anotowanych zdań}
Osadzenia zdań zostały poddane klasteryzacji z wykorzystaniem dwóch algorytmów: HDBSCAN oraz KMeans. Wyniki klasteryzacji przedstawiono w zredukowanej przestrzeni z wykorzystaniem t-SNE. Interaktywne wykresy umożliwiają przeglądanie przypisanych klastrów i przypisanych do nich zdań.

\subsection{Dyskusja wyników}
Wyniki klasteryzacji oraz wizualizacji t-SNE pokazują, że różne modele osadzania zdań (Bert i TF-IDF) generują różne reprezentacje przestrzeni. TF-IDF jest bardziej czuły na częstość występowania słów, natomiast Bert efektywniej modeluje podobieństwa semantyczne, co jest szczególnie widoczne w wynikach klasteryzacji. Na szczególną uwagę zasługuje reprezentacja klastrów dla modelu Bert przy użyciu metody KMeans - zdania występujące w obrębie danych klastrów rzeczywiście dotyczą tych samych tematów.

\end{document}
